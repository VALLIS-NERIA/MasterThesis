\chapter{研究背景}
随着互联网技术的高速发展,如今的互联网有着数据流量大、拓扑结构复杂、业务类型繁多的特点。
传统的网络结构体系在面对如今的互联网时逐渐显得力不从心,主要体现在维护困难、效率低、扩展性和兼容性不佳、安全性不足等。
软件定义网络(Software-Defined Network, SDN)是一种新型网络结构。它的核心思想是将网络中的控制系统与数据转发系统分离开来,形成控制平面和数据平面两个平行的结构。
通过控制平面对网络进行统一管理,SDN可以实现更高的资源利用效率\cite{jain2013b4}\cite{xu2017joint}\cite{xu2018achieving}。
与传统网络中的路由器采用不同的协议来自行选择路由不同,SDN交换机不需要自行计算路由,而是根据控制器下发的流表来对不同的流进行不同的操作。
通过OpenFlow协议,SDN中的控制器可以得知整个网络的拓扑和统计信息,由此在全局层面为不同的流计算其合适的路由,并将路由信息作为流表下发到各个交换机。

为了实现网络的优化,流量统计信息对SDN是至关重要的。统计信息越精确,就越能找出最佳路由方案。然而SDN交换机的硬件资源限制了流量统计的能力。SDN交换机中主要有三种重要的硬件资源。
\textbf{第一种资源是三态内容寻址存储器(Ternary Content Addressable Memory, TCAM)。}TCAM可以快速地根据内容进行查找,因此交换机中往往使用TCAM存储流表。
然而TCAM造价昂贵且非常耗电,导致交换机中的流表项条目数经常很有限。比如HP 5406zl型号交换机只能容纳1500条流表项\cite{curtis2011devoflow}。
\textbf{第二种资源是SRAM。}交换机中的SRAM扮演着主存储器的角色。受成本因素影响,中低端交换机的SRAM大小往往也很有限。
如Juniper入门款交换机仅仅有32MB的SRAM空间\cite{ResourceMonitoring},并且这仅有的SRAM要由包括防火墙、路由控制等多种功能共享。
\textbf{第三种资源是CPU的计算资源。}SDN交换机的主要功能是转发数据包,因此绝大多数的交换机都配备有专门的转发芯片以实现更高的吞吐率。
SDN交换机中的CPU则负责转发数据包之外的计算任务,因此其计算能力通常不高。市面上的许多交换机的CPU主频甚至不足1GHz\cite{wang2014scotch}。

在SDN当中进行流量统计,目前主流有三种方法。第一种方法是使用交换机中的TCAM流表来进行统计。根据OpenFlow协议\cite{pfaff2012openflow},每个流表项当中都有一个字段用来统计符合该流表项的总流量。
然而如前所述,TCAM限制了流表项的条目数。即使是比较高端的Boradcom Trident2型号,也最多只支持1.6万个条目\cite{cohen2014effect}。而数据中心的流数动辄上百万\cite{kandula2009nature},1.6万个条目显然远远不够。
当网络中的流数超出流表上限时,常用的解决方法是将多条流通过规则整合的方式整合到一个流表项中\cite{zhao2018joint}。在这种情况下,流表项中统计的是符合该规则的所有流的流量,从而导致其中单个流的流量不得而知。

第二种方法是通过采样的方式进行统计,比如当前大多数交换机所支持的NetFlow\cite{estan2004building}或sFlow\cite{phaal2004sflow}解决方案。
然而,采样统计的测量精度常常不尽如人意\cite{yu2013software}\cite{li2016flowradar}。
提高采样统计的精度的最直接方法就是提高采样频率,但提高采样频率意味着消耗更多的计算和内存资源,从而影响在高数据量网络中的可伸缩性。

第三种方法是使用称为“Sketch”的数据结构进行流量统计。在实际场景中,往往并不需要得知所有流的流量统计信息,而只关心其中一部分特殊流的信息。
通过特殊的算法,Sketch可以利用很少的内存空间来统计所有数据包,并且只留下关注的信息。
目前已经有很多种不同的Sketch被设计出来,分别适用于不同的使用场景,并在测量精确度和资源占用中取得平衡\cite{KXW06}\cite{li2012per}\cite{estan2002new}。

对于SDN中的很多应用,网络中流量最大的若干条流的流量是最为重要的(也被称为top-$k$问题)。例如,在实际场景中,控制器经常会向交换机部署默认路径以提升新流的响应速度。
然而,当许多流携带着大量流量同时经由默认路径转发时,网络中有可能会出现拥堵。解决此问题的一种方案是,从这些通过默认路径的流当中找出其中流量较大的一些流,对它们进行重路由,从而实现更好的负载均衡。
因此,在交换机的层面,得知大流的较为准确的流量统计数据是非常重要的。

在此背景下,设计一种能够测量大流流量的高效的Sketch是很有必要的。其中设计的重点在于“高效”,也就是在不牺牲太多测量精度的前提下,尽量地减少Sketch本身的计算负载。
如前文所述,由于交换机的CPU性能较弱,且还需要负责执行OpenFlow控制协议以及其它维持交换机运行的必要功能,因此可分配给流量测量的计算资源极其有限。
如\cite{curtis2011devoflow}中的测量结果所示,即使在没有流量出入的情况下,测试用的交换机每秒只能完成275条流表项的设置。
假设包括Sketch在内的其它应用占用了50\%的CPU时间,那么每秒能处理的流表项数量只有137条,在面对较大流量时会不可避免出现性能瓶颈。

目前现有的一些通用Sketch,如UnivMon\cite{liu2016one},尽管可以获得所需的流量统计信息,但是受存储空间所限,不可能记录下所有的流的ID。
为了获得所需的流的ID,有的方案采取了对数据包头进行编码的方式,但这样又加重了交换机的计算负载。
SketchVisor\cite{huang2017sketchvisor}设计了一个top-$k$的sketch,以及一个能够和其它sketch协作的框架,但是在流量较大的情况下它的计算负载仍然太大。
而如果单独使用它的top-$k$ sketch,在存储未命中的情况下则需要修改内存中的k个计数器,会带来无法接受的计算负载。

CountSketch\cite{charikar2004finding}和Filtered Space-Saving(FSS)\cite{homem2010finding}是两种用来测量top-$k$的sketch。
尽管它们可以分辨出大流的ID并记录这些大流的流量统计信息,但是这两种sketch的计算负载对于交换机而言仍然过大。
例如,CountSketch和FSS都需要在一个堆当中搜索某个给定的流ID是否存在。在不使用辅助数据结构的情况下,这个操作不得不遍历整个堆。具体的计算负载的分析将会在后续章节中分析。%TODO

由于现有的sketch无法在保持统计精度的同时降低自身的计算负载,因此我们有必要设计一个新的测量方法,拥有较高的统计精度而不占用过多的计算和存储资源。
本文提出了名为“CountMax”的轻量级流量测量方案。与SketchVisor、CountSketch和FSS相比,CountMax有着存储占用低、计算负载低、结构简单可分布的优势,并且有着不相上下的流量统计精度。
本文首先提出了CountMax的top-$k$ sketch,并在理论上证明了它的精确度和时间复杂度;随后讨论了CountMax在几种常见的应用中的性能。
实验结果显示,在占用相同存储空间的情况下,CountMax的计算负载只有CountSketch和FSS的二分之一到三分之一。并且由于CountMax节约内存的特点,在相同内存占用时,CountMax的统计精确度略高于CountSketch和FSS。

%最后,本文讨论了如何在不同交换机中部署测量工具的问题
% \begin{enumerate}
%     \item 存储占用低。CountMax不需要任何辅助数据结构,除了几十字节的元数据之外,占用的所有内存都是有效信息。
%     \item 计算负载低。对每个到达的数据包,CountMax的每一“行”只需要执行一次哈希操作、一次比较操作和一次加法操作。
%     \item 结构简单易于分布。一个CountMax sketch可以轻易地被拆分为多个,部署在不同的交换机上。通过安排每个交换机只处理一部分的流,可以将计算负载更平均地均摊到每个交换机之间。
% \end{enumerate}

