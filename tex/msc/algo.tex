\section{MSC问题的贪心解法:GMSC}\label{sec:gmscsec}
本节我们提出一个简单但有效的算法解决MSC问题,称为GMSC。



\subsection{算法设计}\label{sec:gmsc}
如第\ref{sec:mscdef}节中定理\ref{thm:nphard}的证明,在只考虑交换机$v_i$的情况下,MSC问题退化为单背包问题。
我们可以对每个交换机分别求解单背包问题,获得每个交换机的最大容纳价值,记为$p(v_i)$。
接下来选择$p(v_i)$最大的那个交换机,将背包问题的解应用到此交换机上。
最后,由于这个交换机被分配了掩码,要相应的更新这些掩码在其余交换机上的价值。
重复这个过程,直到所有交换机都被分配过为止。

这是一个贪心(Greedy)算法,因而称其为GMSC。
\begin{algorithm}[htb]
    \small
    \SetAlgoLined
    \KwData{$V$:交换机的集合;$F$:覆盖的所有流的集合}
    $F=\Phi$\;
    \While{$|V|>0$}{
        \ForEach{$V$中的交换机$v$}{
            求解单背包问题,得到$v$的最大价值$p(v)$和对应的解$S(v)$。记$S(v)$中覆盖的流的集合为$F(v)$\;
        }
        找出$m$,使$p(v_m)$最大\;
        将$S(v_m)$中的掩码应用到$v_m$上\;
        将$F(v)$中覆盖的流加入$F$\;
        从$V$中移除$V_m$\;

        \ForEach{$V$中的交换机$v_i$}{
            \ForEach{$\Omega$中的掩码$r_j$}{
                更新$p(\Pi_i^j)$,去除$\Pi_i^j \cap F(v)$部分的价值\;
            }
        }
    }
    \caption{GMSC}
    \label{alg:gmsc}
\end{algorithm}

\subsection{单背包问题的解法}
作为一个经典问题,单背包问题有多种不同的解法,其中最经典的是动态规划法 \cite{martello1999dynamic}和贪心法\cite{cmulec10}。
这两种解法早已写入算法教科书中,因而在此不再赘述。
设背包容量为$B$,可选择的物品数为$m$,则动态规划法的时间复杂度是$O(m\cdot B)$,近似比为1;
贪心法的时间复杂度是$O(m \log{m})$,近似比为1/2。

\subsection{GMSC的性能分析}
\subsubsection{近似性能}
\begin{theorem}\label{tm:gmscappr}
    GMSC的近似比为$\frac{\mu}{1+\mu}$,其中$\mu$为单背包问题解法的近似比。
\end{theorem}

\begin{proof}
令$Q_G$表示通过GMSC所覆盖的流,$G_l$表示GMSC在第$l$次迭代后所覆盖的流,则$Q_G = G_n$。
令$X_l$代表第$l$次迭代中所选择的掩码带来的价值增量,则:
\begin{equation}
X_l =w(G_l \backslash \bigcup\nolimits^{l-1}_{i=1}G_i)
\end{equation}

考虑GMSC已经完成$l-1$轮迭代的情况。在第$l$轮中,GMSC选择了交换机$v^l$和其上的流$X_l$。
作为对比,最优解法也一定会从$v^l$中选择一个掩码的集合,其中包含的流的集合记为$O_l$。
设最优解法最终覆盖的流的集合为$OPT$,则$OPT = \bigcup\nolimits_{l=1}^{n}O_l$。
如果在这一轮,我们选择$O_l$而不是$X_l$的话,则这一轮的价值增量为$w(O_l \backslash \bigcup\nolimits^{l-1}_{i=1}G_i)$,记为${X'}_l$。

设在已完成$l-1$轮的情况下,$v^l$上的背包问题的最优解是$Y_l$,则根据近似比的定义有$X_l \ge \mu Y_l$。
又因为$Y_l$是当前情况下的短视最优解,故$Y_l \ge {X'}_l$。
因而我们有:

\begin{align}
    X_l &\ge \mu Y_l \notag\\
        &\ge \mu {X'}_l \notag\\
        &= \mu \cdot w(O_l \backslash \bigcup\nolimits^{l-1}_{i=1}G_i) \notag\\
        &\ge \mu \cdot w(O_l \backslash Q_G) \notag
\end{align}

根据$Q_D$和$OPT$的定义,有:
\begin{align}
    w(Q_D)&=\sum\nolimits_{l=1}^{n} X_l \notag\\
        &\ge \sum\nolimits_{l=1}^{n} \mu \cdot w(O_l \backslash Q_D) \notag\\
        &=  \mu \cdot \sum\nolimits_{l=1}^{n} w(O_l \backslash Q_D) \notag\\
        &\ge \mu \cdot w ((\bigcup\nolimits_{l=1}^{n}O_l) \backslash Q_D) \notag\\
        &=  \mu \cdot w( OPT \backslash Q_D) \notag\\
        &\ge \mu \cdot [ w(OPT)-w(Q_D)] \notag
\end{align}

因而:
\begin{equation}\label{eq:gmscappr}
    w(Q_D) \ge \frac{\mu }{1+\mu} \cdot w(OPT)
\end{equation}
定理\ref{tm:gmscappr}得证。
\end{proof}

由定理\ref{tm:gmscappr}我们可以很容易得出:使用动态规划解背包问题时,GMSC的近似比为1/2;使用贪心法时,GMSC的近似比为1/3。

\subsubsection{时间复杂度}
若单背包问题的解法的时间复杂度为$O(\rho)$,更新一个掩码的价值的时间复杂度为$O(\tau)$,则GMSC的时间复杂度为$O(n^2\cdot m \cdot \tau + n^2 \cdot \rho)$。

假设一个掩码中最多有$h$个流,那么$\tau = h\cdot m$。因而:

\begin{theorem}\label{tm:gmsctime}
    GMSC的时间复杂度为$O(n^2\cdot m^2 \cdot h+ n^2 \cdot \rho)$。
\end{theorem}