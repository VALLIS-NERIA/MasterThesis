\chapter{流量统计设施的部署}
在前几章,我们使用CountMax解决了数据平面上流量统计的获取问题,接下来我们从控制平面探讨流量统计该如何部署。

\section{部署问题的必要性}
第\ref{sec:coop}节提出了一个协作式的简单的部署策略,将计算负载均摊给所有入口和出口交换机。
但是在现实网络中的情况更加复杂,简单的协作式很可能不能满足需求。
例如,基于第\ref{sec:observation}节的假设和第\ref{sec:proto}节的测试结果,当$\beta=0.6$的时候,CountMax会占用72\%的CPU时间。
但如果交换机的CPU性能较弱,比如只有单核1GHz,这个数字就会变成216\%。
显然,试图通过优化sketch算法的计算负载是几乎无法解决这个巨大矛盾的。
协作式方法可以最多降低50\%的计算负载,但即使这样,CPU占用依然有108\%。

然而,协作式方法提供了一个优化的方向,即让交换机只处理一部分的流的数据包,而非全部经过交换机的数据包。
在很多分层拓扑结构(如Fat-tree \cite{al2008scalable}和Spine-Leaf \cite{alizadeh2013data})中,所有流都按照接入层——汇聚层——接入层,至少经过3个交换机进行路由。
第\ref{sec:coop}节中的协作式方法只利用起了链路上的2个交换机,如果能利用到更多的交换机,就可以进一步减少单个交换机的计算负载。
因此,我们需要在控制平面上研究更加完善的流量统计设施的部署。

\section{部署问题与现有问题的关系}
流统计部署问题与一个现有问题非常相似,即流统计收集问题(Flow Statistics Colletion, FSC)。
\subsection{流统计收集问题(FSC)}
FSC问题的背景是:所有交换机都统计了经过自己的所有流的统计信息,控制器要有选择地收集这些信息,以免造成过大的控制链路负载和信息延迟。
例如,在一个中等规模的数据中心网络中\cite{kannan2013compact},一个连接着40个服务器的交换机每分钟经过的流可能会有7.5到10万条。
如果要将所有统计信息都上报的话,首先根据DevoFlow \cite{curtis2011devoflow}中的研究,收集统计信息会对交换机的性能产生较大影响。
其次,按照第\ref{subsec:memory}节中的分析,即使上报的报文中只包含五元组的流ID和流量大小,其大小也有约1.5到2.2MB。
如果加入流的起止时间、数据包数,则报文大小会达到3到4MB。假设控制链路的带宽是100Mbps,传输报文需要约0.5秒的时间。
在控制器方面,若网络中有50台交换机,每进行一次统计信息的收集就会产生约200MB的数据,控制器需要对这些数据全部进行处理。
显然,由于一条流会经过多个交换机,这些数据当中包含了大量的重复信息。
如果能减少重复信息的上报,就能够大幅降低交换机、控制器以及控制链路的负载。

FSC问题主要有三种解决方案:\emph{per-flow} \cite{van2014opennetmon}、\emph{per-switch} \cite{su2014flowcover}\cite{su2015cemon}\cite{tootoonchian2010opentm}和基于掩码(wildcard)的方案\cite{xu2017miniming}。
其中掩码方案可以适应最广泛的情形。
即:为每个交换机下发一系列规则,当交换机收到控制器的流统计收集指令时,只上报符合这些规则的流的信息。

\subsection{FSC与流统计部署问题的关系}
FSC与流统计部署问题之间的区别在于,FSC是先统计全部的流,再按照规则上报;流统计部署问题则是先设置好规则,只统计符合规则的流。
因而FSC问题的解决方案稍作修改即可用于解决流统计部署问题。
如:事先为每个交换机设置规则,只有符合规则的流才会被测量,在上报的时候上报所有被测量的流的信息。
