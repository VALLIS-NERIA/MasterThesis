

\section{算法设计}
首先我们确定CountMax算法设计的目标,然后再来介绍算法的具体设计。
CountMax的算法设计包含两部分,其一为统计信息的记录和更新,其二为统计信息的查询。

\subsection{设计目标}
根据第\ref{chap:sketch}章中对已有的sketch方案的分析,为了解决已有sketch的不足,CountMax的设计目标有以下几点:
\begin{enumerate}
	\item 大流识别。在实际应用中我们很难获得关于大流的先验知识,因此CountMax必须负责将大流的ID从成千上万的流当中识别出来。
	\item 更低的计算负载。CountMax应当占用尽可能少的CPU资源,一方面可以应对更重的网络负载,另一方面也为交换机上的其它应用留出了空间。
	\item 较低的测量误差。CountMax对大流流量的估算误差应当可控,且处于可接受的较低范围内。
	\item 追踪更多的大流。在消耗同样的存储空间的前提下,被CountMax所追踪的大流越多,就可以提供越多的有用信息,提升重路由等应用的质量。
\end{enumerate}

\subsection{设计简述}
为了尽量降低计算负载,CountMax摒弃了提高复杂度的辅助数据结构。我们选择仿照Count-Min的设计,只使用bitmap存储数据。
同时,为了识别大流,CountMax在bitmap的每一个格子中都增加了一个存放流ID的字段。
当发生哈希冲突时,Count-Min不对流做区分,全部累加到计数器当中。这样的做法导致了Count-Min无法记录大流的ID。
因此,在CountMax的设计中,在多条流发生哈希冲突时,会根据当前流的ID和bitmap中现有的流ID的关系来选择不同的操作。
可简要描述为:若二者相同,则计数器增加;若二者不同,则计数器减去相应的大小。若减去后的结果为负数,对应格子中的流ID字段也会更改为当前流的ID。
由于网络中的大流的大小往往是小流的上千倍,当一条大流和多个小流碰撞时,这样的算法可以在绝大多数情况下保证最终记录下来的流ID是大流的ID。
而对于大流之间的哈希碰撞,则可以通过增加bitmap行数的方式降低其概率。

\subsection{统计信息记录}
为了记录统计信息,我们维护一个有$d$行、$w$列的bitmap。
Bitmap中的每个格子包含两个字段:$Key$和$Counter$。参数$d$和$w$的意义将在后文讨论。
和bitmap一起的还有$d$个不同的哈希函数$\{h_{1},...,h_{d}\}$。
每个哈希函数$h_j$都能将任意流的ID(如五元组)单向映射为一个介于$0$和$w-1$之间的整数:
\begin{equation}
\notag h_j: \{...\} \to \{0, ..., w-1\}
\end{equation}

\begin{algorithm}[htb]
    \small
    \SetAlgoLined
    \KwData{$f_i$:流的ID; $c_i$:数据包的大小}
  
    \For{$j$从0到$d-1$}{
        $u_j = h_{j}(f_i)$\;
        $f' = Key_{j}[u_j]$\;
        \eIf{$f_i == f'$}{
            $Counter_{j}[u_j] = Counter_{j}[u_j]+c_i$\;
        }{
            \If{$Counter_{j}[u_j] > c_i$}{
                $Counter_{j}[u_j] = Counter_{j}[u_j]-c_i$\;
            }{
                $Counter_{j}[u_j] = c_i-Counter_{j}[u_j]$\;
                $Key_{j}[u_j] = f$\;
            }
        }
    }
    \caption{CountMax对每个数据包的处理过程}
    \label{alg:count_max}
\end{algorithm}
当交换机接收到第$i$号数据包时,首先从数据包中解析出流的ID,记为$f_i$。对于bitmap中的第$j$行,先计算$f_i$的哈希值$u_{j}=h_{j}(f_i)$作为索引。
如果$Key_{j}[u_{j}]$与$f_i$相同,那么对应的$Counter_{j}[u_{j}]$就会加上数据包的大小,记做$c_i$。否则比较$Counter_{j}[u_{j}]$和$c_i$之间的大小。
如果$Counter_{j}[u_{j}]$\textbf{大于}$c_i$,那么$Counter_{j}[u_{j}]$就减去$c_i$。写成表达式即为$Counter_{j}[u_{j}] = Counter_{j}[u_{j}] - c_i$。
如果$Counter_{j}[u_{j}]$\textbf{小于}$c_i$,则令$Counter_{j}[u_{j}]=c_i - Counter_{j}[u_{j}]$,并且将$Key_{j}[u_{j}]$替换为$f_i$。

对bitmap中的每一行迭代上述过程。算法\ref{alg:count_max}详细描述了这一算法。

\subsection{流量信息查询}
%\input{query.tex}
要查询流$f_i$的统计信息,首先从每一行中分别查询。对于sketch中的第$j$行,采用如下的方法获取$f_i$在当前行内的估计$\hat{a}_{i_{j}}$:
\begin{equation}
\hat{a}_{i_{j}}=\left\{
\begin{aligned}
Counter_{j}[h_j(f_i)]&\text{, 若 $f_i=Key_{j}[h_j(f_i)]$}\\
0&\text{, 若 $f_i \ne Key_{j}[h_j(f_i)]$}
\end{aligned}
\right.
\end{equation}

要推断流的流量大小,我们用上述方法查询每一行,从得到的结果中选取最大值作为估计值。
\begin{equation}
\label{eq:query}
\hat{a}_{i}=\max{\{\hat{a}_{i_{j}} , \forall j\}}
\end{equation}

\section{算法分析}\label{sec:analysis}
\label{subsec:analysis}
%\input{analysis.tex}

CountMax具有找出大流的能力。在分析CountMax的具体性能之前,首先对“大流”进行定义。
在网络中,随着流量分布的变化,字面意义上的top-$k$,即网络中流量最大的$k$条流的特征很难量化。因此,我们采用“heavy hitter”作为大流的定义。

给定网络中的所有流的集合,其中总共有$n$条流,所有流的流量之和是$F$。给定一个参数$\delta$ ($0<\delta \le 1$),则每一条流量大于或等于$\delta\cdot F$的流就称为这个集合中的$\delta$-``heavy hitter"。

\subsection{近似性能分析}
本小节将证明,在一定概率内,CountMax可以对heavy hitter的估计值有着可控的近似比。

\begin{theorem}
	\label{tm:query}
    对每条$\delta$-heavy hitter流$f_i$,令$\hat{a}_i$和$a_i$分别表示$f_i$的估算流量和实际流量。
    令$\tilde{d}$表示bitmap中记录有$f_i$的行的数目(即满足$Key[h(f_i)] = f_i$的行数),$e$代表自然对数的底,如下不等式始终满足:
	\begin{equation}
	\label{eq:hhacc}
	Pr[1-\frac{e}{w\cdot \delta}\le \frac{\hat{a}_i}{a_i} \le 1] \ge 1-e^{-\tilde{d}}
	\end{equation}
\end{theorem}

\begin{proof}
首先,由CountMax的性质易知,$\hat{a}_i$始终不大于$a_i$,即:
\begin{equation}
    \hat{a}_i/{a_i} \le 1
\end{equation}

接下来引入变量$I_{i,k,j}$来指示$f_i$和$f_k$两条流是否在第$j$行发生哈希冲突。即:
\begin{equation}
    I_{i,k,j}=\left\{
    \begin{aligned}
    1&\text{, if ($f_i \ne f_k $ ) $\land $ ($h_j(f_i)=h_j(f_k)$)}\\
    0&\text{, otherwise}
    \end{aligned}
    \right.
\end{equation}
由于哈希函数在理论上是独立的,因此对$f_i$和$f_k$的哈希计算可视为两个独立实验。又因为对不同的输入,理想的哈希函数将其散列到每个格子的概率是均等的,因此有:
\begin{equation}
    E(I_{i,k,j})=Pr[h_j(f_i)=h_j(f_k)] = \frac{1}{w}
\end{equation}

接下来定义变量$X_{i,j}$,令其表示在第$j$行中和$f_i$产生了哈希冲突的所有流的流量之和。即:
\begin{equation}\label{eq:xij}
    X_{i,j}=\sum\nolimits_{k=1}^{n}I_{i,k,j}\cdot a_k
\end{equation}

由于当$Key[h_j(f_i)]\ne f_i$时,查询的结果是0,因此必须找到$Key[h_j(f_i)] = f_i$的充分条件。
对于第$j$行,如果 $a_i > X_{i,j}$,也就是$a_i$大于这个格子所处理的总流量的一半,根据鸽巢原理,无论数据包的到达顺序如何,$Key[h_j(f_i)]$ 都必然是 $f_i$,并且$Counter[h_j(i)]\ge a_i-X_{i,j}$。
也就是说,对于$f_i$而言,$\tilde{d}$大于等于满足$a_i > X_{i,j}$的行的数量。
另外,如果$a_i \le X_{i,j}$,因为$Counter[h_j(i)]$永远不会为负,因此$Counter[h_j(i)]\ge a_i-X_{i,j}$依旧成立。

\begin{equation}\label{eq:ai_xij}
    Counter[h_j(i)]\ge a_i-X_{i,j}
\end{equation}

根据等式\eqref{eq:xij}可得:
\begin{equation}\label{eq:ex}\notag
E(X_{i,j})=E(\sum_{k=1}^{n}I_{i,k,j}\cdot a_k)\le\sum_{k=1}^{n} a_k\cdot E(I_{i,k,j})\le \frac{1}{w}\sum_{k=1}^{n}a_k.
\end{equation}


再根据$\delta$的定义,若$f_i$是$\delta$-heavy hitter,则网络中的总流量不超过$\frac{a_i}{\delta}$。因此:
\begin{equation}\label{eq:ex-delta}
E(X_{i,j})\le \frac{1}{w}\sum_{k=1}^{n}a_k\le \frac{1}{w}\cdot \frac{a_i}{\delta}
\end{equation}

综合以上结论,可以进行如下的推导。为方便起见,“$\forall j$”在这里代表“ $\forall j\in \{j|Key[h_j(f_i)] = f_i\}$”。
\begin{align}\notag
&Pr[\hat{a}_i< a_i-\frac{e}{w}\cdot\frac{a_i}{\delta}]\\\notag
&=Pr[\forall j, Counter[h_j(f_i)]<a_i-\frac{e}{w}\cdot\frac{a_i}{\delta} ]\\\notag
&\le Pr[\forall j, a_i-X_{i,j} <a_i-\frac{e}{w}\cdot\frac{a_i}{\delta} ] \text{ (根据不等式\eqref{eq:ai_xij})}\\\notag
&=Pr[\forall j, X_{i,j} > \frac{e}{w}\cdot\frac{a_i}{\delta} ]\\\notag
&\le Pr[\forall j, X_{i,j}>e\cdot E(X_{i,j})]\text{\ \ (根据马尔可夫不等式)}\\\notag
&<e^{-\tilde{d}}
\end{align}
	
将该不等式左侧的条件取反,可得结论:
\begin{equation}
Pr[\frac{\hat{a}_i}{a_i} \ge 1-\frac{e}{w\cdot \delta}]\ge 1-e^{-\tilde{d}}
\end{equation}

定理\ref{tm:query}得证。
\end{proof}


接下来分析$\tilde{d}$的数学期望。

\begin{theorem}\label{tm:acc}
对任何$\delta$-heavy hitter,$E[\tilde{d}]\ge d\cdot(1-1/(w\cdot\delta))$。
\end{theorem}

\begin{proof}
对于$\delta$-heavy hitter流 $f_i$以及bitmap中的第$j$行,根据上一节的推断,如果$a_i > X_{i,j}$那么$Key[h(f_i)] $ 一定为 $f_i$。

根据马卡洛夫不等式和不等式\eqref{eq:ex-delta},可得:
\begin{align}\notag
	Pr[X_{i,j}>a_i] &\le \frac{E[X_{i,j}]}{a_i}\\\notag
	&\le \frac{1}{w}\cdot\frac{a_i}{\delta}\cdot\frac{1}{a_i}\\
	&\le \frac{1}{w\cdot\delta}
\end{align}
因此 $Pr[Key[h_j(f_i)]=f_i]\ge 1- Pr[X_{i,j}>a_i] \ge 1-\frac{1}{w\cdot\delta}$。
    
由于bitmap中的每一行是独立的,因此$E[\tilde{d}]\ge d\cdot(1-1/(w\cdot\delta))$。
\end{proof}

\subsection{时间复杂度分析}
\begin{theorem}\label{tm:time}
CountMax对一个数据包进行处理的过程的时间复杂度是$O(d)$。
\end{theorem}

\begin{proof}
由于每一行的操作是固定的一次哈希、一次比较、一次加法,因而每行的时间复杂度是$O(1)$。
CountMax总共有$d$行,所以时间复杂度是$O(d)$。
\end{proof}



